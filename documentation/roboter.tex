\section{Visualisierung des Manutec r2 unter ROS}

\subsection{Installieren der Nodes}
Um das Projekt einzubinden müssen die folgenden Kommandos der Reihe nach ausgeführt werden. \\ \\
Zu erst geht man in den richtigen Workspace, welcher der Catkin Workspace ist. 

\begin{lstlisting}[language=bash]
cd ~/catkin_ws/src/
\end{lstlisting}
Anschließend wird das Projekt aus Github kopiert und mit folgendem Befehl eingefügt (Tipp: Man kann den Link auch einfach aus der Dokumentation kopieren)
\begin{lstlisting}[language=bash]
git clone https://github.com/informatik-mannheim/virtual-robot-ros.git
\end{lstlisting}
Um das Projekt zu kompilieren muss man den unten ausgeführten Befehl ausführen. Dies muss man auch bei jeder Änderung machen um das Projekt zu aktualisieren.
\begin{lstlisting}[language=bash]
cd ~/catkin_ws/
\end{lstlisting}

\begin{lstlisting}[language=bash]
catkin_make
\end{lstlisting}

\subsection{Starten der Nodes}
Um das Meta-Betriebssystem ROS zu starten wird folgender Befehl benötigt
\begin{lstlisting}[language=bash]
roscore
\end{lstlisting}
Starten des TCP-Server Nodes:
\begin{lstlisting}[language=bash]
rosnode ...
\end{lstlisting}
Starten der Manutec r2 Simulation:
\begin{lstlisting}[language=bash]
roslaunch ...
\end{lstlisting}



\subsection{Simulation des Manutec r2 Roboters}
Um das Projekt zu simulieren wird die Codesys Datei ausgeführt. Diese startet automatisch auch das ROS und die Launch-Datei, die die Bewegungen des Roboters simuliert. 

\subsection{Beenden der Nodes}
Zum beenden der Nodes, einfach mit \textit{STRG+C} den aktuellen Node schließen.


